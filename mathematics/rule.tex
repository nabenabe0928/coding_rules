\documentclass[dvipdfmx]{article}
\usepackage[utf8]{inputenc}
\usepackage{amsmath,amssymb,bm,bbm}
\usepackage{graphicx}
\usepackage{ascmac}
\usepackage{fancyhdr}
\usepackage{graphicx}
\usepackage{algorithm, algpseudocode}
\usepackage{algpseudocode}
\usepackage{tikz}
\usepackage{ulem}
\usepackage{booktabs}
\usepackage{multirow}
\usepackage[caption=false]{subfig}
\usepackage{comment}
\usepackage{listings}
\usetikzlibrary{chains}
\usetikzlibrary{calc}
\usepackage{mathtools}
\usepackage{amsmath,amsthm,tikz}
\usepackage{lastpage}
\usepackage{tcolorbox}
\tcbuselibrary{breakable, skins, theorems}
\newtheorem{theorem1}{Theorem}
\newtheorem{theorem2}{Definition}
\newtheorem{theorem3}{Assumption}
\def\qed{\hfill $\Box$}

% remove the end from algorithm
\algtext*{EndFor}
\algtext*{EndWhile}
\algtext*{EndIf}
\algtext*{EndProcedure}
\algtext*{EndFunction}
\newcommand{\Break}{\textbf{break}}
\newcommand{\Continue}{\textbf{continue}}

\setlength{\textwidth}{156mm}
\setlength{\textheight}{220mm}
\setlength{\oddsidemargin}{5mm}
\setlength{\voffset}{-15mm}
\setlength{\headsep}{10mm}
\pagestyle{fancy}
\rhead{\thepage/\pageref{LastPage}}
\cfoot{ }

\lhead[Math writing rules / Shuhei Watanabe]{Math writing rules / Shuhei Watanabe}

\title{\vspace{-15mm} Math writing rules}
\author{Shuhei Watanabe}
\date{\today}

\begin{document}
\maketitle
\thispagestyle{fancy}

\section{Introduction}
\begin{proof}
  This is a proof.
\end{proof}

Center-aligned equations
\begin{gather}
  f(x) = x^2 + 2x + 1 \\
  g(x) = 2x + 1
\end{gather}

\&-aligned equations
\begin{equation}
\begin{aligned}
  f(x) &= x^2 + 2x + 1 \\
  g(x) &= 2x + 1
\end{aligned}
\end{equation}

Mathematical abbreviations (xxx v.s. $\backslash$xxx)
\begin{itemize}
  \item $cos, det, dim, exp, inf, lim, log, min, max, sup$
  \item $\cos, \det, \dim, \exp, \inf, \lim, \log, \min, \max, \sup$
\end{itemize}

Mathematical abbreviations ($\backslash$text\{xxx\}y v.s. $\backslash$xxx y)
\begin{equation}
  \begin{aligned}
    \text{log} x, \text{lim}_{x \rightarrow \infty}x
  \end{aligned}
\end{equation}
\begin{equation}
  \begin{aligned}
    \log x, \lim_{x \rightarrow \infty} x
  \end{aligned}
\end{equation}
Alternative: $\backslash$text\{xxx\} $\tilde{}$ y

Fraction in sentences or exponential
\begin{itemize}
  \item \textbf{Bad looking}: $\frac{1}{2}, 2^{\frac{1}{2}}$
  \item \textbf{Better looking}: $1 / 2, 2^{1 / 2}$
\end{itemize}

Center dots v.s. bottom dots
\begin{equation}
\begin{aligned}
  \text{Good}:~& x_1, x_2, \dots, x_n \\
  \text{Bad}:~& x_1, x_2, \cdots, x_n
\end{aligned}
\end{equation}
\begin{equation}
  \begin{aligned}
    \text{Good}:~& \sum_{i=1}^n x_i = x_1 + x_2 + \cdots x_n,
    \prod_{i=1}^n x_i = x_1 x_2 \cdots x_n \\
    \text{Bad}:~& \sum_{i=1}^n x_i = x_1 + x_2 + \dots x_n,
    \prod_{i=1}^n x_i = x_1 x_2 \dots x_n \\
  \end{aligned}
  \end{equation}

  \begin{theorem1}\leavevmode \par
    This is a theorem about Eq.~(1).
  \end{theorem1}
  \begin{theorem1}\leavevmode \par
    This is a theorem about Eq.~$\mathrm{(1)}$.
  \end{theorem1}

Transpose
\begin{itemize}
  \item $A^T$
  \item $A^\top$
\end{itemize}

Vectors
\begin{itemize}
  \item $\boldsymbol{v}$
  \item $v$
\end{itemize}

Brackets
\begin{itemize}
  \item normal: $(x)$
  \item big: $\bigl(x\bigr)$
  \item Big: $\Bigl(x\Bigr)$
  \item bigg: $\biggl(x\biggr)$
  \item Bigg: $\Biggl(x\Biggr)$
\end{itemize}

norm
\begin{itemize}
  \item $\|\boldsymbol{x}\|$
  \item $||\boldsymbol{x}||$
\end{itemize}

empty set
\begin{itemize}
  \item $\phi$
  \item $\emptyset$
\end{itemize}

underbrace with mathclap (mathtools)
\begin{equation}
\begin{aligned}
  a + \underbrace{b}_{\text{long long long}} + c \\
  a + \underbrace{b}_{\mathclap{\text{long long long}}} + c \\
\end{aligned}
\end{equation}

Union and intersections
\begin{equation}
\begin{aligned}
  \cup_i S_i, \cap_i S_i \\
  \bigcup_i S_i, \bigcap_i S_i \\
\end{aligned}
\end{equation}

For all, for some in sentences
\begin{itemize}
  \item we apply $f$ for all $i$
  \item we apply $f, \forall i$ 
\end{itemize}

Mapping of sets and correspondence of elements
\begin{equation}
\begin{aligned}
  \mathbb{R}^n &\rightarrow \mathbb{R} \\
  \boldsymbol{x} &\mapsto f(\boldsymbol{x})
\end{aligned}
\end{equation}

Definition
\begin{equation}
\begin{aligned}
  f(x) &\coloneqq x^2 \\
  f(x) &:= x^2 \\
\end{aligned}
\end{equation}

\section{Mathematics}

A specific value $a$ belongs to a set $A$
\begin{equation}
\begin{aligned}
  a \in A
\end{aligned}
\end{equation}

A subset $S$ of a set $A$
\begin{equation}
\begin{aligned}
  S \subset A
\end{aligned}
\end{equation}

A set
\begin{equation}
\begin{aligned}
  \text{A set of positive integers}:&~ \mathbb{Z}_{+} = \{1, 2, \dots\} \\
  \text{A set of $x$ s.t. $x > 0$}:&~ \mathbb{R}_{+} = \{x \mid x > 0 \} \\
  \text{A set of $x = y^2$ s.t. $y \in \mathbb{R}$}:&~ \mathbb{R}_{\geq 0} = \{y^2 \mid y \in \mathbb{R} \}
\end{aligned}
\end{equation}

A complement set $A^c$
\begin{equation}
\begin{aligned}
  A \cap A^c = \emptyset
\end{aligned}
\end{equation}

Cartesian product
\begin{equation}
\begin{aligned}
  X \times Y \coloneqq \{(x, y)\mid x\in X, y \in Y\}
\end{aligned}
\end{equation}

Booleans
\begin{itemize}
  \item $x^2$ is non negative for an arbitrary real number $x$: $\forall x \in \mathbb{R}, x^2 \geq 0$ 
  \item There exist some real numbers $x$ such that $x$ is negative: $\exists x \in \mathbb{R}, x < 0$ 
  \item For all positive $\epsilon$, there exist some numbers $x$ such that $|f(x)| < \epsilon$:
  $\forall \epsilon > 0, \exists x, \text{ s.t. } |f(x)|<\epsilon$
\end{itemize}

Derivatives at a given point
\begin{equation}
\begin{aligned}
  \biggl(\frac{dy}{dx}\biggr)_{x = a}
\end{aligned}
\end{equation}

Norm
\begin{equation}
\begin{aligned}
  \|\cdot\|&:~ X \rightarrow [0, \infty),
  \boldsymbol{x} \mapsto \|\boldsymbol{x}\| \\
  |\cdot|&:~ \mathbb{R} \rightarrow [0, \infty),
  x \mapsto |x|
\end{aligned}
\end{equation}
Use norm for vectors or 
Norm  absolute value

subscription, superscription (optima)
Avoid superscription as much as possible.
If use superscriptions, try to avoid confusion
by non-iteric, a word or bracket
Use mostly subscription

pipe, semicolon
pipe is for conditioning and
semicolon is for fixed parameters.

inner product
\begin{equation}
\begin{aligned}
  \langle x, y \rangle
\end{aligned}
\end{equation}

a sequence
\begin{equation}
\begin{aligned}
  \{a_i\}_{i \in [0, N) \subset \mathbb{Z}}
  = \{a_i\}_{i=0}^{N - 1}
\end{aligned}
\end{equation}

hat, bar, tilde, direction by subscription


\bibliographystyle{splncs04}
\bibliography{ref}
\end{document}